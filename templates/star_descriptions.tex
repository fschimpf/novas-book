43 - Acrux ist der hellste Stern im Sternbild Kreuz des Südens (Crux) und der zwölfthellste des Himmels. Da er sehr weit im Süden liegt, erhielt er in den europäischen Kulturen keinen Eigennamen; die Bezeichnung Acrux ist einfach aus Alpha und Crux gebildet. In der Astronomie wird er systematisch als α Crucis oder kurz α Cru bezeichnet, entsprechend folgen Becrux und Gacrux.

Mit einer scheinbaren Helligkeit von 0,77 mag ist er der südlichste Stern erster Größe. 


44 - Gacrux (Kunstwort aus „Gamma“ und „Crux“) ist der Name des Sterns γ Crucis (Gamma Crucis) im Sternbild Kreuz des Südens, dessen dritthellster Stern er ist. Gacrux ist nur von der südlichen Hemisphäre sichtbar. Sein Name dürfte durch den Astronomen Elijah Hinsdale Burritt (1794–1838) durch Zusammenziehung seiner Bayer-Bezeichnung eingeführt worden sein.

Mit einer scheinbaren Helligkeit von 1,59 mag ist Gacrux der 25-hellste Sterne des Firmaments. Gacrux ist ein Roter Riese der Spektralklasse M3.5. Seine Entfernung von der Sonne beträgt etwa 90 Lichtjahre (Hipparcos Datenbank), seine absolute Helligkeit ca. −0,6 mag. Die Ausmaße des Roten Riesen werden auf etwa 3 Sonnenmassen, 110 Sonnendurchmesser und (bei Berücksichtigung seiner Infrarot-Strahlung) 1500-fache Sonnenleuchtkraft geschätzt. 


46 - Alioth (von arabisch ألية, DMG alya ‚Fettschwanz [des Schafs], Gesäß‘) ist der Eigenname des Sterns Epsilon Ursae Maioris (ε UMa) im Sternbild Großer Bär. Er ist der dem Wagenkasten nächste der drei Deichselsterne des Großen Wagens. Alioth besitzt eine scheinbare Helligkeit von 1,8 mag und ist etwa 83 Lichtjahre entfernt. Er ist geringfügig heller als Dubhe und Alkaid und damit der hellste Stern im Sternbild sowie auch insgesamt einer der 50 hellsten Sterne am Nachthimmel. 



49 - Spica (lateinisch Kornähre), auch α Virginis, Azimech oder Alaraph genannt, ist der hellste Stern im Sternbild Jungfrau und der fünfzehnthellste Stern am nächtlichen Sternenhimmel.

Im Sternbild Jungfrau stellt Spica die Kornähre in der Hand der Jungfrau dar. In Mesopotamien war Spica unter dšala šubultu (Gott/Göttin Schala, die Kornähre), später auch Göttliche Jungfrau der Kornähre, bekannt. Die Römer glaubten, in Spica die Göttin Ceres zu sehen. Römische Namen sind Spicum, Spigha, Stachys (von gr. stakhum Kornähre) und Aristae Puella (Kornmädchen).

Die IAU hat am 30. Juni 2016 den Eigennamen Spica als standardisierten Eigennamen festgelegt.[1] 



50 - Alkaid (von arabisch القائد, DMG al-qāʾid ‚der Anführer‘) ist der Eigenname des Sterns Eta Ursae Maioris (η UMa) im Sternbild Großer Bär. Er steht am vorderen Ende der Deichsel des Großen Wagens. Alkaid besitzt eine scheinbare Helligkeit von 1,9 mag und ist ca. 104 Lichtjahre entfernt. Er gehört zu den 50 hellsten Sternen am Nachthimmel.

Es handelt sich bei Alkaid um einen blauweißen Hauptreihenstern von 6-facher Masse, 4-fachem Durchmesser und 460-facher Leuchtkraft der Sonne. Seine Oberflächentemperatur liegt bei 15700 K. Der Stern rotiert ungewöhnlich schnell, für ihn wurde eine projizierte äquatoriale Rotationsgeschwindigkeit v∙sin i von 195 km/s gemessen. Durch seine rasche Rotation ist er geschätzt um etwa 5 % abgeplattet.[11]

Alkaid ist ein Veränderlicher vom Typ 53 Persei. Dies sind langsam pulsierende B-Sterne (SPB) von 3–9 Sonnenmassen. Er zeigt ganz geringfügige Helligkeitsschwankung um 0,01 mag mit einer Periode von etwa 2,8 Tagen.

Anders als die meisten anderen Sterne des Großen Wagens gehört er nicht zur sogenannten Bärengruppe, einem Bewegungshaufen von über einhundert Sternen. Der Stern bewegt sich für Beobachter auf der Erde mit einer schnellen Eigenbewegung von etwa 122 Millibogensekunden/Jahr über den Himmel, fast in die entgegengesetzte Richtung wie bei den Sternen der Bärengruppe. Bei seiner Entfernung entspricht dies einer Geschwindigkeit von etwa 18 km/s, während er sich zusätzlich mit einer Geschwindigkeit von 13 km/s auf uns zu bewegt. Im Raum bewegt sich der Stern demnach mit einer Geschwindigkeit von etwa 23 km/s relativ zu unserer Sonne.

In der alt-arabischen Deutung wurden die drei Deichselsterne des Großen Wagens als بنات النعش / banāt an-naʿš /‚Töchter der Totenbahre‘ angesehen, d. h. Klageweiber, die vor der Totenbahre hergehen. Der vorderste der Sterne war der قائد بنات النعش / qāʾid banāt an-naʿš /‚Anführer der Klageweiber‘. In der traditionellen europäischen Astronomie wurde der Stern davon abgeleitet Alkaid oder auch Benetnasch genannt. Nach dem „IAU Catalog of Star Names“ der Working Group on Star Names (WGSN) der IAU zur Standardisierung von Sternnamen wurde im Jahr 2016 dem Stern η Ursae Maioris offiziell der Name „Alkaid“ zugewiesen.[12] 



51 - Beta Centauri (abgekürzt β Cen) ist das zweithellste stellare Objekt der markanten Konstellation Centaurus am Südhimmel (0,6 mag). Es handelt sich hier um ein Sternsystem, bestehend aus einem spektroskopischen Doppelstern (A) und einem Einzelstern (B). Komponente A wird oft als Einzelobjekt betrachtet, weshalb das System oft als Doppelstern (AB) bezeichnet wird. Es hat neben der Bayer-Bezeichnung β (= zweithellster im Sternbild) auch den offiziellen Eigennamen Hadar, wird aber auch Agena (lat. für Knie (des Zentauren)), genannt.

Das System ist etwa 530 Lichtjahre vom Sonnensystem entfernt und wurde lange für einen einzelnen blau-weißen Überriesen gehalten, dessen Leuchtkraft jene der Sonne um mindestens das 10.000-fache übertrifft. Erst 1935 konnte ihn J.G. Voute als Doppelstern enttarnen, dessen Komponenten (A, B) nur 1,3" Winkelabstand haben. Trotzdem ändert sich der Positionswinkel nur langsam, sodass die Umlaufzeit rund 300 Jahre betragen muss. Der kleinere Stern (Hadar B) hat die Helligkeit 4,1 mag und strahlt 1500-mal heller als die Sonne, wird aber vom helleren Zentralgestirn fast überstrahlt. 




53 - Arktur oder Arcturus (altgriechisch Ἀρκτοῦρος Arktúros, α Bootis, englisch manchmal α Boötis) ist der Hauptstern im Bärenhüter (Bootes), einem auffälligen Sternbild am Frühlingshimmel. Arktur ist der hellste Stern des Nordhimmels und der dritthellste am gesamten Sternhimmel. Nur Sirius und der von Mitteleuropa aus nicht sichtbare Canopus strahlen heller, gehören aber zum Südhimmel. Arktur ist von allen Kontinenten aus zu sehen (mit Ausnahme der inneren Antarktis) und war wahrscheinlich der erste Stern, der mit einem Teleskop am Taghimmel beobachtet wurde (1635 durch Jean-Baptiste Morin).[10] Man findet ihn leicht in der Verlängerung der Deichsel des Großen Wagens. Wenn man den gebogenen Sternenzug über Arcturus hinaus in die gleiche Richtung noch weiter verlängert, gelangt man zu Spica.

Nach Messungen durch den Astrometriesatelliten Hipparcos ist Arktur 36,7 Lichtjahre (11,3 Parsec) von der Erde entfernt, also astronomisch gesehen relativ nahe. Arktur befindet sich, wie auch die Sonne, derzeit in der Lokalen Flocke. Hipparcos’ Beobachtungen deuten auch darauf hin, dass Arktur ein Doppelstern sein könnte. Alle bisherigen Versuche, einen Begleiter nachzuweisen, sind jedoch gescheitert oder haben ein negatives Resultat geliefert. Die Auflösung eines möglichen Begleiters liegt momentan an der Grenze des technisch Möglichen; es ist gegenwärtig keine abschließende Aussage über seine Existenz möglich.[10]

Arktur bildet zusammen mit den anderen Alphasternen Spica (Jungfrau) und Regulus (Löwe) das Frühlingsdreieck; ersetzt man den Regulus durch Denebola, die Schwanzspitze des Löwen, wird daraus ein annähernd gleichseitiges. Ein kleineres, beinahe gleichseitiges Dreieck bildet er mit zwei schwächeren Sternen in der Umgebung seiner Sichtlinie: Seginus (γ Bootis) und Gemma (α CrB), siehe Sternkarte rechts. 

Die hohe Eigenbewegung von Arktur ist bemerkenswert. Sie ist höher als die aller anderen Sterne erster Magnitude der Sternennachbarschaft (ausgenommen Alpha Centauri) und wurde das erste Mal 1718 von Edmond Halley (1656–1742) festgestellt. Arktur bewegt sich mit einer Geschwindigkeit von etwa 122 km/s relativ zum Sonnensystem. Gemeinsam scheint er sich in einer Gruppe von 52 anderen Sternen, die auch die „Arkturgruppe“ genannt wird, zu bewegen.

Arktur ist jetzt fast auf seinem sonnennächsten Punkt, den er in ca. 4000 Jahren erreichen wird. In dieser Zeit wird er seinen Abstand nur noch um etwa 0,1 % verringern. Er ist erst seit ca. einer halben Mio. Jahren mit freiem Auge sichtbar und wird in etwa der gleichen Zeit für das freie Auge wieder unsichtbar, wenn er seine Reise in seiner eigenen Umlaufbahn um die Milchstraße fortsetzt. 



54 - Alpha Centauri [ˈalfa t͡sɛnˈtaʊ̯ʀi] (α Centauri, abgekürzt α Cen, aber auch Rigil Kentaurus, Rigilkent, Toliman oder Bungula genannt) ist im Sternbild des Zentauren am Südhimmel ein etwa 4,34 Lichtjahre entferntes Doppelsternsystem. Es bildet zusammen mit dem ihn umkreisenden, 0,21 Lj von Alpha Centauri entfernten sonnennächsten Roten Zwerg Proxima Centauri (etwa 4,2465 Lj Abstand zur Sonne) ein hierarchisches Dreifachsternsystem.[9] Alpha Centauri besteht aus dem helleren gelben Stern Alpha Centauri A und dem orangefarbenen Alpha Centauri B in derzeit 6″ Abstand. Zusammen mit der Sonne befindet es sich in der sogenannten Lokalen Flocke. Nur 4,4° westlich steht mit Beta Centauri ein weiterer Stern 1. Größe.

Als teleskopischer (nur im Fernrohr trennbarer) Doppelstern ist Alpha Centauri mit einer scheinbaren Gesamthelligkeit von −0,27 mag das hellste Objekt im Sternbild und der dritthellste Stern am Nachthimmel. Der hellere Alpha Centauri A alleine hat eine scheinbare Helligkeit von −0,01 mag und ist damit der vierthellste Stern am Himmel.[10] 



56 - Zuben-el-dschenubi oder Zubenelgenubi (aus arab. الزبانى الجنوبي az-zubānā al-ğanūbī ‚die südliche Schlange‘) ist die Bezeichnung des Sterns Alpha Librae (α Librae) im Sternbild Waage in einer Entfernung von 76 Lichtjahren. Zuben-el-dschenubi ist ein spektroskopischer Doppelstern bestehend aus zwei weißen Sternen des Spektraltyps A3.[5] Er hat eine scheinbare Helligkeit von +2,75 mag und eine absolute Helligkeit von 0,92 mag. Zuben-el-dschenubi besitzt einen Begleitstern der scheinbaren Helligkeit +5,15 mag, der wegen seines großen Winkelabstands von 231" (= 3'51") schon im Feldstecher aufgefunden werden kann (Positionswinkel 314°). Der Spektraltyp Begleiters ist F3 V, seine absolute Helligkeit +3,34 mag.

Zuben-el-dschenubi kann als ekliptiknaher Stern vom Mond und von Planeten bedeckt werden. Die letzte Bedeckung von Zuben-el-dschenubi durch einen Planeten erfolgte am 25. Oktober 1947 durch die Venus, die nächste, wegen geringer Sonnenelongation nur sehr schwer beobachtbare Bedeckung von Zuben-el-dschenubi durch einen Planeten wird am 10. November 2052 durch den Planeten Merkur erfolgen.

Die IAU hat am 21. August 2016 den Eigennamen Zubenelgenubi als standardisierten Eigennamen festgelegt, allerdings nur für den Hauptstern α2.[6] Der Begleitstern α1 trägt demnach nicht diesen Eigennamen. 



57 - Kochab (von arab.: الكوكب, DMG al-kaukab ‚der Stern‘) oder β Ursae Minoris (Beta Ursae Minoris, kurz β UMi) ist die Bezeichnung des zweithellsten Sterns im Sternbild Kleiner Bär (Kleiner Wagen). Ursprünglich hieß er الكوكب الشمالي al-kaukab asch-schamālī ‚der nördliche Stern‘ und galt arabischen Astronomen als Polarstern. Inzwischen ist er aufgrund der Präzession der Erde aus dieser Position gerückt.

Kochab ist mit einer scheinbaren Helligkeit von 2,1 mag mit bloßem Auge gut zu erkennen und bildet zusammen mit Pherkad den Abschluss des Kastens des „Kleinen Wagens“. Er ist auf der Nordhalbkugel bis 16° nördlicher Breite als Zirkumpolarstern zu sehen.

Er ist ein orangeroter Riesenstern und gehört der Spektralklasse K4 und der Leuchtkraftklasse III an. Seine Entfernung von der Sonne beträgt etwa 130 Lichtjahre. Kochab gilt als Einzelstern. 



59 - α Coronae Borealis (kurz α CrB) ist ein etwa 76,5[5] Lichtjahre von der Erde entferntes Doppelstern-System und der hellste Stern im halbkreisförmigen Sternbild Corona Borealis (Nördliche Krone). Es handelt sich um ein bedeckungsveränderliches System vom Typ Algol. Dieser Stern wird unter anderem auch als Gemma (lateinisch: Edelstein), Alphekka oder Alphecca (aus dem Arabischen) bezeichnet.

Die IAU hat am 20. Juli 2016 den Eigennamen Alphecca als standardisierten Eigennamen festgelegt.[6]

Die Hauptkomponente ist ein Hauptreihenstern der Spektralklasse A0 (Oberflächentemperatur von etwa 9500 Kelvin) mit etwa der 60-fachen Leuchtkraft der Sonne, einer Masse von 2,6 Sonnenmassen und einem Radius von rund 3 Sonnenradien. Der Begleiter ist ein G-Hauptreihenstern mit etwa 0,9 Sonnenmassen.

Die Helligkeit des Systems schwankt mit einer Periode von 17,36 Tagen[4] zwischen etwa 2,2 mag und 2,3 mag (mit dem bloßen Auge kaum feststellbar).

Die Eigenbewegung lässt annehmen, dass α Coronae Borealis dem Ursa-Major-Strom (Bärenstrom) angehört, einem am Himmel weit verstreuten Bewegungshaufen, von dessen Hauptsternen im Großen Wagen sie 30–40° entfernt steht.

Um 1895 wurde aus Anomalien im Linienspektrum vermutet, dass Gemma ein Mehrfachstern sein könnte. Neuere Messungen haben das aber nicht bestätigt.[7] 



61 - Antares (Silbentrennung: Ant-ares), auch Alpha Scorpii (α Scorpii) genannt, ist der hellste Stern im Sternbild Skorpion. Er ist etwa 600 Lichtjahre von der Erde entfernt. Der Name stammt von altgriechisch ἀντί antí, deutsch ‚gegen‘, und dem Namen Ἄρης Ares und bedeutet „Gegenares“ (Gegenmars). Der Gott Ares wurde von den Römern Mars genannt. Antares hat eine ähnliche Farbe wie Mars, weshalb beide leicht zu verwechseln sind, zumal sie sich stets in der Nähe der Ekliptik befinden und phasenweise eine ähnliche Helligkeit besitzen. Weitere Namen sind Qalbu l-ʿAqrab (arabisch قلب العقرب, Herz des Skorpions) , An ta er (mandarin 安塔尔) und Vespertilio. 



62 - Atria ist der Eigenname des Sterns α Trianguli Australis (Alpha Trianguli Australis). Der Name ist ein Kunstwort aus der Bayer-Bezeichnung (Alpha Trianguli Australis). Atria gehört der Spektralklasse K2 an und ist etwa 400 Lichtjahre entfernt. 



64 - Lambda Scorpii (auch Shaula, bzw. Alascha) ist ein Mehrfachsternsystem im Sternbild Skorpion. Lambda Scorpii befindet sich am Stachel des Skorpions, der Name (arabisch الشولاء, DMG aš-šaulāʾ) bedeutet „erhobener Schwanz“. Das System hat eine scheinbare Helligkeit von +1,62 mag, womit es zu den 50 hellsten Sternen am Nachthimmel gehört. Es ist ca. 600 Lichtjahre entfernt (Hipparcos Datenbank) und Mitglied des Gouldschen Gürtels. 


65 - Ras Alhague (von arab. رأس الحية raʾs al-ḥayya "Kopf der Schlange"), auch Rasalhague oder α Ophiuchi (Alpha Ophiuchi), ist ein Stern im Sternbild Schlangenträger mit einer scheinbaren visuellen Helligkeit von +2,1 mag. Der Stern bildet die Nordspitze des Sternbildes und befindet sich etwa 5° südöstlich des Sterns Ras Algethi (α Herculis). 


67 - Etamin, auch Eltanin oder Ettanin, ist der Eigenname des Sterns γ Draconis (Gamma Draconis, kurz γ Dra).

Etamin gehört der Spektralklasse K5 an und besitzt eine scheinbare Helligkeit von 2,2 mag. Etamin ist ca. 150 Lichtjahre entfernt.

Der Name Etamin und seine überlieferten Varianten Eltanin und Ettanin bedeuten „Seeungeheuer“ oder „Schlange“ (der arabische Name des Sternbildes). Ein anderer Name von Etamin ist Alnath und ist bedeutungsgleich mit dem Namen von β Tauri, El Nath (arab. „das Ende“). Ein weiterer Name von Etamin ist Rastaban, was vom arabischen ar-rās at-tinnīn / الراس التنين abgeleitet ist und in etwa „der Kopf der Schlange (des Drachen)“ bedeutet. 


68 - Kaus Australis (arabisch قوس, DMG Qaus ‚Bogen‘ und lateinisch australis, „südlich“), Bayer-Bezeichnung Epsilon Sagittarii, ist ein Doppelstern mit einer scheinbaren visuellen Helligkeit von 1,79 mag im Sternbild Schütze, womit er zu den 50 hellsten Sternen am Nachthimmel gehört. 



69 - Wega, auch Vega, oder in der Bayer-Bezeichnung α Lyrae, ist der Hauptstern des Sternbildes Leier (Lyra). Der Name leitet sich vom arabischen Ausdruck النسر الواقع, an-nasr al-wāqiʿ ab, was in Übersetzung „herabstoßender (Adler)“ bedeutet. Der Stern ist Teil des großen Sommerdreiecks und im weißen Licht der hellste Stern des Nordhimmels. Mit seiner Magnitude von 0,0 diente er früher als Referenzstern der Helligkeitsmessung (Fotometrie). Wega befindet sich, wie auch die Sonne, in der Lokalen Flocke. 



71 - Altair [al'ta:ir] (auch Atair genannt) ist der hellste Stern im Sternbild Aquila (Adler) und der zwölfthellste Stern am Nachthimmel, Bayer-Klassifizierung α Aquilae. Zusammen mit den Sternen Wega und Deneb bildet Altair das Sommerdreieck. Altair befindet sich, wie auch die Sonne, derzeit in der Lokalen Flocke. Mit einer Entfernung von 16,73 Lichtjahren gehört er unter den Sternen, die mit bloßem Auge zu sehen sind, zu denjenigen, die der Sonne am nächsten liegen. 



72 - α Pavonis (Alpha Pavonis) ist der hellste Stern im Sternbild Pfau. Er besitzt eine scheinbare Helligkeit von 1,94 mag und ist nur südlich des 32. Grades nördlicher Breite sichtbar. Er gehört zur Spektralklasse B2 IV, die ihn als blauen Unterriesen charakterisiert. Seine Entfernung beträgt ca. 180 Lichtjahre. α Pavonis ist ein spektroskopischer Doppelstern mit einer Umlaufperiode von 11,8 Tagen, woraus man auf eine Entfernung der beiden Komponenten von bloß 0,21 AE schließen kann.

α Pavonis wird auch Peacock genannt. Dies ist keine klassische Bezeichnung, sondern der Name ist modernen Ursprungs: Er wurde α Pavonis in den späten 1930er Jahren verliehen, als ein Navigationsalmanach für die Royal Air Force erstellt wurde. Von 57 Sternen des neuen Almanachs hatten zwei – ε Carinae und α Pavonis – keine überlieferten Namen, so dass für sie neue Bezeichnungen – Avior und (nach dem Sternbild) Peacock (englisch "Pfau") – erfunden wurden.[4]

Die IAU hat am 20. Juli 2016 den Eigennamen Peacock als standardisierten Eigennamen für diesen Stern festgelegt.[5] 



73 - Deneb ist der hellste Stern (α Cygni) im Sternbild Schwan. Er bildet zusammen mit Wega und Altair das Sommerdreieck. Mit einer scheinbaren Helligkeit von 1,2 mag ist der Alpha-Cygni-Stern Prototyp Leuchtkräftiger Blauer Veränderlicher. Deneb ist der 19.-hellste Stern am Nachthimmel und zusammen mit Eta Carinae (ebenfalls LBV-Stern) der hellste bekannte Stern unserer Milchstraße im sichtbaren Licht. Zudem ist er der entfernteste Stern 1. Größenklasse. Hätte Deneb den gleichen Abstand zur Erde wie Wega (25 Lichtjahre), würde er annähernd so hell wie der Mond in Sichelform leuchten.[5] 
Deneb ist ein heißer Überriese, er befindet sich momentan in der Übergangsphase vom Blauen Riesen zum Roten Überriesen. Er ist also gerade dabei, von der Hauptreihe nach rechts oben hin abzubiegen, d. h., er wird röter (die Temperatur sinkt), aber gleichzeitig leuchtkräftiger, da sich sein Durchmesser vergrößert. Mit einer absoluten Helligkeit von −8,5 mag gehört er zu den hellsten bekannten Sternen. Die Strahlungsleistung beträgt etwa 1,2 × 1032 W und ist damit rund 300.000-mal höher als die der Sonne.[2] Da die Entfernung noch nicht genau bestimmt werden konnte, schwanken die Werte der Helligkeit zwischen 100.000- und 250.000-facher Sonnenhelligkeit.[4] Der Stern erzeugt in einer Minute mehr Licht als die Sonne in einem Monat. 



75 - Enif (aus arabisch أنف, DMG anf ‚Nase (des Pferdes)‘) ist die Bezeichnung von ε Pegasi (Epsilon Pegasi), des hellsten Sterns im Sternbild Pegasus. Enif hat eine scheinbare Helligkeit von +2,38 mag. Er ist ein orangeroter Überriese vom Spektraltyp K2 und ca. 700 Lichtjahre entfernt. In wenigen Millionen Jahren dürfte er als massereicher Weißer Zwerg enden oder – falls er für diesen Evolutionsweg gerade etwas zu viel Masse besitzt – als Supernova explodieren. Enif ist ein Irregulärer Veränderlicher, dessen Helligkeit manchmal durch enorme Eruptionen (Flares) deutlich zunimmt. 1972 zeigte der Stern einen Helligkeitsausbruch, wobei er mit 0,70m auffallend hell wurde. Enif ist ein Dreifachstern. 



76 - Alnair (Schreibweise der IAU),[1] auch Alpha Gruis, α Gruis oder α Gru (Bayer-Bezeichnung), ist ein etwa 100 Lichtjahre entfernter Stern am Südhimmel im Sternbild Kranich (Grus) am Südhimmel. Alnair ist ein Hauptreihenstern des Spektraltyps B6 und gehört mit einer scheinbaren Helligkeit von 1,7 mag zu den 50 hellsten Sternen am Nachthimmel.

Der Eigenname Alnair geht auf den arabischen Sternnamen Al Nair „der Helle“ zurück. 



78 - Fomalhaut [fom alhˈaɔ̯t][7][8] (α Piscis Austrini) ist der hellste Stern im Sternbild Südlicher Fisch und der 19. in der Liste der hellsten Sterne am Himmel. Der Name bedeutet „Maul des Fisches“ (arab. فم الحوت fam al-ḥūt). Andere Namen: Difda al Auwel, Hastorang, Os Piscis Meridiani.

Fomalhaut befindet sich, wie auch die Sonne, derzeit in der Lokalen Flocke. 



80 - Alpha Pegasi ist ein Stern im Sternbild Pegasus. Er ist etwa 133 Lichtjahre entfernt und hat eine scheinbare visuelle Helligkeit von 2,5 mag.

Der Stern trägt den Eigennamen Markab (oder "Mirfak", arab. "Schulter"). Der Name Markab stammt ursprünglich aus arabisch مركب, DMG Markab ‚Pferdesattel‘, durch falsche Transkription wurde daraus Mankib, welches sich aus arabisch منكب الفرس, DMG Mankib al-Faras ‚Schulter (des Pegasus)‘ herleitet. 











